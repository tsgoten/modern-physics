\documentclass[11pt, twocolumn]{article}
\author{Tarang Srivastava}
\usepackage{amsmath, amsthm, multicol}
\usepackage{graphicx}
\usepackage[margin=.25in]{geometry}
\setlength{\columnsep}{.5in}
\newcommand{\makechaptertitle}[1]{
\begin{center}
	\begin{large}
		#1
	\end{large}
	\begin{small}
		\\by Tarang Srivastava
	\end{small}
\end{center}
}
\newcommand{\dif}[2]{\frac{d#1}{d#2}}
\newcommand{\diff}[3]{\frac{d^#3 #1}{d #2^#3}}
\newcommand{\scinot}[2]{
	#1 \times 10^{#2}
}
\newcommand{\scinotu}[3]{
	#1 \times 10^{#2}{\mathrm{#3}}
}
\newcommand{\wave}{\psi (x,t)}
\theoremstyle{definition}
\newtheorem{q}{}
\newtheorem{ans}{}
\begin{document}
	\makechaptertitle{Chapter 6-8}
	\textbf{Questions}
	\begin{q}
		Express the wave function as an integral of finding a particle in all of space.
	\end{q}
	\begin{q}
		State the expected form of Schrodinger's equation.
	\end{q}
	\begin{q}
		Derive the time independent Schrodinger Equation
	\end{q}
	\begin{q}
		What is the Hamiltonian
	\end{q}
	\begin{q}
		Arrive at the expression of momentum as an operator
	\end{q}
	\begin{q}
		Derive the time independent Schrodinger Equation
	\end{q}
	\begin{q}
		State Schrodinger's equation in Three Dimensions
	\end{q}
	\begin{q}
		Use the time dependent Schrodinger's equation to express the wave function as a second order differential equation of the form 
		\[ f'' + x^2f = 0 \]
	\end{q}
	\begin{q}
		State Heisenberg's Uncertainty Relation
	\end{q}
	\begin{q}
		Arrive at the Time-Energy Uncertainty Relation
	\end{q}
	\begin{q}
		Wave function at the presence of a barrier. Consider some potential barrier in Region II. Region I is before the potential barrier and Region III is after the potential barrier.\\
		State the Schrodinger Equation in each Region and the form of the wave function.
	\end{q}
	\begin{q}
		Solve for the reflection and transmission coefficient.
	\end{q}
	\begin{q}
		Consider a beam of electrons traveling to the right along the $ x$-axis with energy $ E $. The potential energy is $ V = 0 $ for $ x < 0 $, but as $ x = 0 $ there is a potential step, and the potential energy increases to $ V_0 $ for $ x > 0 $. Assuming $ E > V_0 $ (a) calculate the reflection and transmission coefficients, and (b) show that flux is conserved. 
	\end{q}

	

\end{document}