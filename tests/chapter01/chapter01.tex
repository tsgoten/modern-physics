\documentclass[10pt, twocolumn]{article}
\author{Tarang Srivastava}
\usepackage{amsmath, amsthm}
\usepackage{graphicx}
\usepackage[margin=.20in]{geometry}
\newcommand{\makechaptertitle}[1]{
\begin{center}
	\begin{large}
		#1
	\end{large}
	\begin{small}
		\\by Tarang Srivastava
	\end{small}
\end{center}
}
\theoremstyle{definition}
\newtheorem{q}{}
\begin{document}
		\makechaptertitle{Relativity Test}
		\begin{q}
			At what speed does a clock move if it runs at a rate which is one-half the rate of a clock at rest?
		\end{q}
		\begin{q}
			At what speed does a meter stick move if its length is observed to shrink to $ 0.5 $ m?
		\end{q}
		\begin{q}
			The average lifetime of a $ \pi $ meson in its own frame of reference is $ 26.0 $ ns. (this is its proper lifetime)
			\begin{itemize}
				\item if the $ \pi $ meson moves with speed $ 0.95 c $ with respect to the Earth, what is its lifetime as measured by an observer at rest on Earth?
				\item What is the average distance it travels before decaying as measured by an observer at rest on Earth?
			\end{itemize}
		\end{q}
		\begin{q}
			An atomic clock is placed in a jet airplane. The clock measures a time interval of $ 3600 $s when the jet moves with speed $ 500 $ m/s. How much larges a time interval does an identical clock held by an observer at rest on the ground measure?
		\end{q}
		\begin{q}
			A rod of length $ L_0 $ moves with speed $ v $ along the horizontal direction. The rod makes an angle $ \theta_0 $ with respect to the $ x' $ axis. 
			\begin{itemize}
				\item Determine the length of the rod as measured by a stationary observer.
				\item Determine the angle $ \theta $ the rod makes with the $ x $ axis.
			\end{itemize}
		\end{q}
		\begin{q}
			Relativistic Doppler Shift 
			\begin{itemize}
				\item How fast and in what direction must galaxy $ A $ be moving if an absorption line found at wavelength $ 550 $nm (green) for a stationary galaxy is shifted to $ 450 $ nm (blue) ("blue shift")?
				\item How fast and in what direction is galaxy $ B $ moving if it shows the same line shifted to $ 700 $nm (red) ("red shift")?
			\end{itemize}
		\end{q}
		\begin{q}
			A stationary observer on Earth observes spaceships $ A $ and $ B $ moving in the same direction toward the Earth. Spaceship $ A $ has speed $ 0.5 $c and spaceship $ B $ has speed $ 0.8 $c. Determine the velocity of spaceship $ A $ as measured by an observer at rest in spaceship $ B $.
		\end{q}
		\begin{q}
			Two rockets of rest length $ L_0 $ are approaching the earth from opposite directions at velocities $ \pm c/2 $. How long does one appear to the other?
		\end{q}
		\begin{q}
			A person comes to you claiming that he/she has invented a microchip $ 1 $ cm square in size which can run at a clock speed of $ 300,000  $GHz.
			Would you invest in this person's company so that he/she can manufacture and market his/her new invention? Explain your answer. 
		\end{q}
		\begin{q}
			Two spaceships approach each other. They are each viewed from Earth as having a speed half that of light. What is their speed relative to each other?
		\end{q}
		\begin{q}
			If you move toward an emitter of yellow light $ (\lambda = 580nm) $ at half the speed of light, what wavelength would you observe? What would be the answer if the emitter moved toward you?
		\end{q}
		\begin{q}
			A beam of muons is injected in a storage ring, a device that uses electromagnetic fields to maintain the muons in uniform circular motion. The ring's radius is $ 60 $m, and the muons are injected with a velocity such that $ \gamma = 15 $. How many revolutions of the ring will an "average" muon make before it decays? The proper lifetime of a muon is $ 2.2 \times 10^{-6} $s.
		\end{q}
\end{document}