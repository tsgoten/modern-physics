\documentclass[11pt, twocolumn]{article}
\author{Tarang Srivastava}
\usepackage{amsmath, amsthm, multicol}
\usepackage{graphicx}
\usepackage[margin=.25in]{geometry}
\setlength{\columnsep}{.5in}
\newcommand{\makechaptertitle}[1]{
\begin{center}
	\begin{large}
		#1
	\end{large}
	\begin{small}
		\\by Tarang Srivastava
	\end{small}
\end{center}
}
\newcommand{\wave}{\psi (x,t)}
\newcommand{\dif}[2]{\frac{d#1}{d#2}}
\newcommand{\diff}[3]{\frac{d^#3 #1}{d #2^#3}}
\newcommand{\scinot}[2]{
	#1 \times 10^{#2}
}
\newcommand{\scinotu}[3]{
	#1 \times 10^{#2}{\mathrm{#3}}
}

\theoremstyle{definition}
\newtheorem{q}{}
\newtheorem{ans}{}
\begin{document}
	\makechaptertitle{Chapter 6-8}
	\textbf{Questions}
	\begin{q}
		Express the wave function as an integral of finding a particle in all of space.
	\end{q}
	\begin{q}
		State the expected form of Schrodinger's equation.
	\end{q}
	\begin{q}
		Derive the time independent Schrodinger Equation
	\end{q}
	\begin{q}
		What is the Hamiltonian
	\end{q}
	\begin{q}
		Arrive at the expression of momentum as an operator
	\end{q}
	\begin{q}
		Derive the time independent Schrodinger Equation
	\end{q}
	\begin{q}
		State Schrodinger's equation in Three Dimensions
	\end{q}
	\begin{q}
		Use the time dependent Schrodinger's equation to express the wave function as a second order differential equation of the form 
		\[ f'' + x^2f = 0 \]
	\end{q}
	\begin{q}
		State Heisenberg's Uncertainty Relation
	\end{q}
	\begin{q}
		Arrive at the Time-Energy Uncertainty Relation
	\end{q}
	\begin{q}
		Wave function at the presence of a barrier. Consider some potential barrier in Region II. Region I is before the potential barrier and Region III is after the potential barrier.\\
		State the Schrodinger Equation in each Region and the form of the wave function.
	\end{q}
	\begin{q}
		Solve for the reflection and transmission coefficient.
	\end{q}
	\begin{q}
		Consider a beam of electrons traveling to the right along the $ x$-axis with energy $ E $. The potential energy is $ V = 0 $ for $ x < 0 $, but as $ x = 0 $ there is a potential step, and the potential energy increases to $ V_0 $ for $ x > 0 $. Assuming $ E > V_0 $ (a) calculate the reflection and transmission coefficients, and (b) show that flux is conserved. 
	\end{q}
	\begin{q}
		Show that 
		\[ |R|^2 + |T|^2 = 1 \]
	\end{q}
	\begin{q}
		Consider a $ 2000 $kg truck moving towards a bump in the road whose center is at $ x = 0 $ and whose height is given by $ 0 $ for $ x<-0.1 $m, $ (0.05m)\cos(5 \pi x)  $ for $ -0.1m < x < +0.1m $. Assuming that the truck moves so slowly that its kinetic energy is insufficient to climb the bump to any height at all, estimate the probability for the truck to tunnel through the bump.
	\end{q}
	\pagebreak
	\textbf{Solutions}
	
	\begin{ans}
		Probability of finding a particle in ALL of space is obviously one. 
		\[ \int\limits_{all space} | \psi(x,t) | ^2 d^3\vec{r}= 1 \]
	\end{ans}
	\begin{ans}
		The wave function $ \wave $ has to follow the linear relation with it's derivative
		\[ \frac{d \wave}{dt} = [some constant] \wave \]
		So we can expect it to be the form of some exponential. 
	\end{ans}
	\begin{ans}
		Based on our expectation of the form of the equation, the wave function is
		\[ \wave = \exp(kx - \omega t) \]
		Where $ k = \dfrac{\lambda}{2 \pi} = \dfrac{h}{2 \pi p} = \dfrac{\hbar}{p}$ and $ \omega = 2 \pi f $ which using $ E = hf $ gives $ \omega = \dfrac{2 \pi E}{h} = \dfrac{E}{\hbar} $
		We then find that expecting a plane wave function gives us 
		\[ \wave = \exp(i\frac{px-Et}{\hbar}) \]
		Taking the first derivative with respect to time and the second derivative with respect to $ x $ we find
		\[ \dif{\wave}{t} =  \frac{-iE}{\hbar} \wave\]
		\[ \diff{\wave}{x}{2} = \frac{p^2}{h^2} \wave\]
		Using $ E = p^2/2m $ for a free particle we can relate the two functions and get the Time Dependent Schrodinger equation
		\[ i\hbar \dif{\wave}{t} = -\frac{h^2}{2m} \diff{}{x}{2} \wave \]
		If we consider some potential $ V(x) = V_0 $, that will be a part of the energy giving us the complete equation
		\[ i\hbar \dif{\wave}{t} = -\frac{h^2}{2m} \diff{}{x}{2} \wave + V(x)\wave\]
	\end{ans}
	\begin{ans}
		The Hamiltonian is the the operator that only deals with the position part of the Schrodinger equation, and describes the position of the wave function. 
		$$ H = -\frac{\hbar^2}{2m} \diff{}{x}{2}  + V(x) $$
	\end{ans}
	\begin{ans}
		From the calculation in Problem 3 we can show that
		\[ -i \hbar \dif{}{x} \wave = p \wave \]
		\[ p = -i\hbar \dif{}{x} \]
	\end{ans}
	\begin{ans}
		We begin with the Time-Independent Schrodinger equation and use a technique called separation of variables for $ \wave $ to express it as a function of just $ x $. \\
		We express the wave function as
		\[ \wave = u(x)T(t) \]
		Then we use this form in the Schrodinger equation
		\[ i\hbar \dif{\wave}{t} = -\frac{\hbar^2}{2m} \diff{}{x}{2} \wave \]
		We then find
		\[ i\hbar \dif{T(t)}{t} u(x) = -\frac{\hbar^2}{2m} \diff{u(x)}{x}{2} T(t)\]
		We can rearrange it as
		\[ i\hbar \dif{T(t)}{t} \frac{1}{T(t)} = -\frac{1}{u(x)}\frac{\hbar^2}{2m} \diff{u(x)}{x}{2} \]
		Notice that each side of the equation sign is dependent on only $ x $ or $ t $. This means that both sides have to be equal no matter the value of $ x $ or $ t $. Which means that both sides must be a constant, represented by $ E $. So, 
		\[ i\hbar \dif{T(t)}{t} \frac{1}{T(t)} = -\frac{1}{u(x)}\frac{\hbar^2}{2m} \diff{u(x)}{x}{2} = E\]
		So the Time-Independent Schrodinger equation is 
		\[ Eu(x) =  -\frac{\hbar^2}{2m} \diff{u(x)}{x}{2}\]
		or
		\[ Eu(x) = Hu(x) \]
		We complete this by solving for $ T(t) $ which is 
		\[ \dif{T(t)}{t} = \frac{-i E}{\hbar} T(t) \]
		So, $ T(t) $ must be
		\[ T(t) = \exp(\frac{-iEt}{\hbar}) \]
		giving us the wave function as
		\[ \wave = u_E(x) \exp(\frac{-iEt}{\hbar}) \]
	\end{ans}
	\begin{ans}
		Deriving in Three Dimensions is not that difficult but with basic understanding of gradient we can get that it is 
		\[ i \hbar \dif{\psi(\vec{r}, t)}{t} = -\frac{\hbar^2}{2m} \nabla^2 \cdot \psi(\vec{r}, t) + V(\vec{r}) \cdot \psi(\vec{r}, t) \]
	\end{ans}
	\begin{ans}
		The Time-Independent Schrodinger Equation gives 
		\[ Eu(x) = -\frac{\hbar^2}{2m} \diff{u(x)}{x}{2}\]
		Which we can express in the form
		\[ f'' + \omega^2 f = 0 \]
		So, we find
		\[ \diff{u(x)}{x}{2} + \frac{p^2}{\hbar^2} u(x) = 0  \]
		Giving us
		\[ \omega = \pm \frac{p}{\hbar} \]
	\end{ans}
	\begin{ans}
		Based on observational results and modifying the Schrodinger Equation we find
		\[ \Delta p \Delta x \leq \dfrac{\hbar}{2} \]
	\end{ans}
	\begin{ans}
		With some 
		\begin{align}
		\begin{array} { c } { \Delta E = v \Delta p \geq v \frac { \hbar / 2 } { \Delta x } = \frac { \hbar / 2 } { ( \Delta x / v ) } = \frac { \hbar / 2 } { \Delta t } } \\ { \Delta E \Delta t \geq \hbar / 2 } \end{array}
		\end{align}
	\end{ans}
	\begin{ans}
		Region I:
		A combination of the initial wave and the part being reflected back. 
		From Problem 8 we know that $ \omega = \pm p/\hbar $, so we state that the wave function must be of the form 
		\[ u(x) = exp(ikx) \]
		where 
		\[ k = p/\hbar \]
		So the wave function in region I must be 
		$$ u(x) = \exp(ikx) + R\exp(-ikx) $$
		For region II since there is a potential so we have
		\[ \diff{u(x)}{x}{2} + \frac{(E-V_0) 2m}{\hbar^2} u(x) \]
		We can set $ q $ as 
		\[ q = \pm \sqrt\frac{(E-V_0) 2m}{\hbar^2} \]
		So we have the function as
		\[ u(x) = A \exp (ikx) + B \exp (-ikx) \]
		For region III since the energy is the same as region I, but there is only the wave being transmitted we have 
		\[ u(x) = T \exp(ikx)\]
	\end{ans}
	\begin{ans}
		\begin{align}
		\begin{array} { c } { R = \frac { i \left( q ^ { 2 } - k ^ { 2 } \right) \sin ( 2 q a ) } { 2 k q \cos ( 2 q a ) - i \left( k ^ { 2 } + q ^ { 2 } \right) \sin ( 2 q a ) } \exp ( - 2 i k a ) } \\ { T = \frac { 2 k q } { 2 k q \cos ( 2 q a ) - i \left( k ^ { 2 } + q ^ { 2 } \right) \sin ( 2 q a ) } \exp ( - 2 i k a ) } \end{array}
		\end{align}
	\end{ans}
	\begin{ans}
		We basically have at the point $ x = 0 $
		\[ u(x) = \exp(ikx) + R \exp(-ikx) \]
		and
		\[ u(x) = T \exp(iqx) \]
		Taking the derivative of each reveals 
		\[ \dif{u(x)}{x} = ik\exp(ikx) - ikR \exp(-ikx) \]
		\[ \dif{u(x)}{x} = ik T \exp(iqx) \]
		For the value of $ x = 0 $ we get 
		\begin{align*}
			u(0) &= 1 + R \\
			&= T \\
			\intertext{and}
			\dif{u(0)}{x} &= ik - ikR \\
			&= iqT
		\end{align*}
		Leaving us with two equations
		\[ T = 1 + R \]
		\[ k - kR = qT \]
		We can rearrange the second one and get
		\begin{align*}
			\frac{k}{q} (1 - R) = T = 1 +R \\
			k - Rk = q + Rq \\
			\frac{k-q}{k+q} = R \\
			\frac{2k}{k+q} = T
		\end{align*}
	\end{ans}
	\begin{ans}
		content...
	\end{ans}
	\begin{ans}
		The probability transmission is only dependent on $ |T|^2 $ so we have 
		\begin{align*}
		\begin{aligned} | T | ^ { 2 } & \cong \exp \left( - \frac { 2 } { \hbar } \int \sqrt { 2 m V ( x ) } d x \right) \\ & = \exp \left( - \frac { 2 \sqrt { 2 ( 2,000 k g ) ( 2,000 k g ) \left( 9.8 m / s ^ { 2 } \right) ( 0.05 m ) } } { 1.05 \times 10 ^ { - 34 } J \cdot s } \right. \end{aligned}
		\end{align*}
		The remaining integral is then given by
		\begin{align*}
		\frac { 1 } { 5 \pi } [ \sin ( 0.5 \pi ) - \sin ( - 0.5 \pi ) ] = \frac { 2 } { 5 \pi } \cong 0.13
		\end{align*}
	\end{ans}
	
	

\end{document}