\documentclass[10pt, twocolumn]{article}
\author{Tarang Srivastava}
\usepackage{amsmath, amsthm}
\usepackage{graphicx}
\usepackage[margin=.25in]{geometry}
\geometry{letterpaper,textwidth=7.3in,hmarginratio=1:1,	textheight=9in,vmarginratio=1:1,heightrounded}
\newcommand{\makechaptertitle}[1]{
\begin{center}
	\begin{large}
		#1
	\end{large}
	\begin{small}
		\\by Tarang Srivastava
	\end{small}
\end{center}
}
\theoremstyle{definition}
\newtheorem{q}{}
\begin{document}
		\makechaptertitle{Relativity Test}
		\begin{q}
			At what speed does a clock move if it runs at a rate which is one-half the rate of a clock at rest?
		\end{q}
		\begin{q}
			At what speed does a meter stick move if its length is observed to shrink to $ 0.5 $ m?
		\end{q}
		\begin{q}
			The average lifetime of a $ \pi $ meson in its own frame of reference is $ 26.0 $ ns. (this is its proper lifetime)
			\begin{itemize}
				\item if the $ \pi $ meson moves with speed $ 0.95 c $ with respect to the Earth, what is its lifetime as measured by an observer at rest on Earth?
				\item What is the average distance it travels before decaying as measured by an observer at rest on Earth?
			\end{itemize}
		\end{q}
		\begin{q}
			An atomic clock is placed in a jet airplane. The clock measures a time interval of $ 3600 $s when the jet moves with speed $ 500 $ m/s. How much larges a time interval does an identical clock held by an observer at rest on the ground measure?
		\end{q}
		\begin{q}
			A rod of length $ L_0 $ moves with speed $ v $ along the horizontal direction. The rod makes an angle $ \theta_0 $ with respect to the $ x' $ axis. 
			\begin{itemize}
				\item Determine the length of the rod as measured by a stationary observer.
				\item Determine the angle $ \theta $ the rod makes with the $ x $ axis.
			\end{itemize}
		\end{q}
\end{document}