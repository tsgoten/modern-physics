\chapter{Waves As Particles and Particles As Waves}
\section{Nature of Photons}
Particles that make up radiation are called photons. \\
Photons travel in the speed of light, as a quanta. The energy is given by
\[ E = pc \]
Energy is also related to the frequency
\[ E = hf \]
where $ h $ is 
\[ h \approx \scinot{6.63}{-34} \mbox{J$ \cdot $ s} \]
Combining the equations we get
\[ p = \frac { E } { c } = \frac { h f } { c } = \frac { h } { \lambda } \]
we can also express it with angular frequency where \[ \omega = 2 \pi f \]
so
\[ E = \hbar \omega \]
where
\[ \hbar = h / 2 \pi \approx 1.05 \times 10 ^ { - 34 } \mathrm { J } \cdot \mathrm { s } \]
is what is usually called Planck's constant. 
\begin{example}
	Suppose that a $ 60 $W bulb radiates light primarily of wavelength $ 1000 $nm. Find the number of photons emitted per second. \\
	We can divide the total energy release per second, Watts, by the energy per photon, to find photons per second.
	\[ n = \frac { 60 \mathrm { W } } { h f } = \frac { 60 \mathrm { W } } { \left( 6.63 \times 10 ^ { - 34 } \mathrm { J } \cdot \mathrm { s } \right) \left( 3 \times 10 ^ { 14 } \mathrm { s } ^ { - 1 } \right) } \]
	$ n  = 3 \times 10 ^ { 20 } \mathrm { photons } / \mathrm { s }$
\end{example}
\section{Photoelectric Effect}
