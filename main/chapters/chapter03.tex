\chapter{Waves As Particles and Particles As }
\putpic{ch4}
\section{Nature of Photons}
Particles that make up radiation are called photons. \\
Photons travel in the speed of light, as a quanta. The energy is given by
\[ E = pc \]
Energy is also related to the frequency
\[ E = hf \]
where $ h $ is 
\[ h \approx \scinot{6.63}{-34} \mbox{J$ \cdot $ s} \]
Combining the equations we get
\[ p = \frac { E } { c } = \frac { h f } { c } = \frac { h } { \lambda } \]
we can also express it with angular frequency where \[ \omega = 2 \pi f \]
so
\[ E = \hbar \omega \]
where
\[ \hbar = h / 2 \pi \approx 1.05 \times 10 ^ { - 34 } \mathrm { J } \cdot \mathrm { s } \]
is what is usually called Planck's constant. 
\begin{example}
	Suppose that a $ 60 $W bulb radiates light primarily of wavelength $ 1000 $nm. Find the number of photons emitted per second. \\
	We can divide the total energy release per second, Watts, by the energy per photon, to find photons per second.
	\[ n = \frac { 60 \mathrm { W } } { h f } = \frac { 60 \mathrm { W } } { \left( 6.63 \times 10 ^ { - 34 } \mathrm { J } \cdot \mathrm { s } \right) \left( 3 \times 10 ^ { 14 } \mathrm { s } ^ { - 1 } \right) } \]
	$ n  = 3 \times 10 ^ { 20 } \mathrm { photons } / \mathrm { s }$
\end{example}
\section{Photoelectric Effect}
Metals contain a large number of free electrons (sea of electrons).The electrons don't leak but when a photon strikes one of the free electrons the photon may under the right circumstance trigger the electron to escape the metal. \\
Electrons emitted by a metal are called \textbf{photoelectrons}. The \textbf{work function} $ W $ of the metal tells the values for which there will be a current. \\
if $ hf $ is larger than $ W $, then the electrons will emerge with speed $ v $ such that
\[ \dfrac{1}{2} m_e v^2 = hf -W \]
The $ KE $ of an electron is independent of intensity. 
\begin{example}
	EM radiation of wavelength $ 270 \mathrm{ nm } $ on aluminum releases photoelectrons. The most energetic of which are stopped by a potential difference of $ 0.406 $ volts. Calculate work function of aluminum in electron volts. \\
	The energy can be found for the electron using
	\[ K = e V = \left( 1.6 \times 10 ^ { - 19 } \mathrm { C } \right) ( 0.406 \mathrm { V } ) \] \[ = 0.65 \times 10 ^ { - 10 } \mathrm { J } \]
	energy for the photon
	\[ E = h f = \frac { h c } { \lambda } = \] \[ \left( 6.63 \times 10 ^ { - 4 } \mathrm { J } \cdot \mathrm { s } \right) \left( 3.00 \times 10 ^ { 8 } \mathrm { m } / \mathrm { s } \right)\] \[ / \left( 270 \times 10 ^ { - 9 } \mathrm { m } \right) = 737 \times 10 ^ { - 19 } \mathrm { J } \]
	the difference is the work function
	\[ W = E - K = 6.72 \times 10 ^ { - 19 } \mathrm { J }\] \[ = \left( 6.72 \times 10 ^ { - 19 } \mathrm { J } \right) / \left( 1.6 \times 10 ^ { - 19 } \mathrm { J } / \mathrm { eV } \right) = 42 \mathrm { eV } \]
\end{example}
\section{The Compton Effect}
Consider collisions between light and free electrons. That is, 
\[ \gamma + e \rightarrow \gamma + e \]
Let the initial electron be at rest with zero momentum and energy 
\[ m_ec^2 \]
Initially, the photon has energy
\[ hf \]
associated with it is a momentum $ \vec{ \mathrm { q } } $ whose initial magnitude is 
\[ | \vec{ \mathrm { q } }| = hf/c \]
After the collision the photon's energy is $ hf' $ and momentum is $ \vec{ \mathrm { q } }' $.\\
The final electron momentum is \[ \vec{ \mathrm { p } } \] and its energy is 
\[ \sqrt { p ^ { 2 } c ^ { 2 } + m _ { e } ^ { 2 } c ^ { 4 } } \]
conservation of momentum gives us
\[ \vec { q } = \vec { q } ^ { \prime } + \vec { p } \]
and
\[ h f + m _ { e } c ^ { 2 } = h f ^ { \prime } + \sqrt { p ^ { 2 } c ^ { 2 } + m _ { e } ^ { 2 } c ^ { 4 } } \]
So the obvious question is, now what does the light \textit{look} like: what's is wavelength $ \lambda' $?\\
First we see
\[ \begin{aligned} p ^ { 2 } & = \left( \vec { q } ^ { \prime } - \vec { q } \right) ^ { 2 } = q ^ { \prime 2 } + q ^ { 2 } - 2 \vec { q } ^ { \prime } \cdot \vec { q } \\ & = \left( \frac { h f ^ { \prime } } { c } \right) ^ { 2 } + \left( \frac { h f } { c } \right) ^ { 2 } - 2 \left( \frac { h f ^ { \prime } } { c } \right) \left( \frac { h f } { c } \right) \cos \theta \end{aligned} \]
where $ \theta $ is the angle between the initial and final photon momentum vectors. After rearranging we find 
\[ \begin{aligned} p ^ { 2 } + m _ { e } ^ { 2 } c ^ { 2 } & = \left( \frac { h f } { c } - \frac { h f ^ { \prime } } { c } + m _ { s } c \right) ^ { 2 } \\ & = \left( \frac { h f } { c } \right) ^ { 2 } + \left( \frac { h f ^ { \prime } } { c } \right) ^ { 2 } - 2 \left( \frac { h f } { c } \right) \left( \frac { h f ^ { \prime } } { c } \right) \\ & + 2 m _ { e } h \left( f - f ^ { \prime } \right) + m _ { e } ^ { 2 } c ^ { 2 } \end{aligned} \]
substituting in values for $ p^2 $
\[ 2 \left( \frac { h f } { c } \right) \left( \frac { h f ^ { \prime } } { c } \right) ( 1 - \cos \theta ) = 2 m _ { e } h c \left( \frac { f } { c } - \frac { f ^ { \prime } } { c } \right) \]
By using the general relation $ f = c/\lambda $, we easily find that
\[\lambda ^ { \prime } - \lambda = \frac { h } { m _ { e } c } ( 1 - \cos \theta )  \]
This expression tells the \textit{change} $ \Delta \lambda $ after scattering. \\
The parameter 
\[ \frac { h } { m _ { e } c } \]
obviously results in the dimension of length and is called the \textbf{Compton wavelength of the electron}. It's magnitude is
\[ \scinotu{2.4}{-12}{m} \]
\begin{example}
	X ray of $ \lambda = \scinotu{5.53}{-2}{nm} $ is scatter and detected at an angle of $ 35^ \circ $. Find the fractional shift of the wavelength. \\
	\small
	\[ \begin{aligned} \frac { \lambda ^ { \prime } - \lambda } { \lambda } = \frac { h } { m _ { e } c \lambda } ( 1 - \cos \theta ) \\  = \frac { \left( 6.63 \times 10 ^ { - 34 } \mathrm { J } \cdot \mathrm { s } \right) \left( 1 - \cos \left( 35 ^ { \circ } \right) \right) } { \left( 0.91 \times 10 ^ { - 30 } \mathrm { kg } \right) \left( 3.00 \times 10 ^ { 8 } \mathrm { m } / \mathrm { s } \right) \left( 5.53 \times 10 ^ { - 11 } \mathrm { m } \right) } \\ = 7.9 \times 10 ^ { - 3 } \text { . } \end{aligned} \]
	\normalfont
	around a $ 1\% $ shift 
\end{example}
\section{Blackbody Radiation}
What is a blackbody radiation?\\
Black-body radiation is the thermal electromagnetic radiation within or surrounding a body in thermodynamic equilibrium with its environment, or emitted by a black body
Okay...what's a blackbody then?\\
blackbody is an idealized physical body that absorbs all incident electromagnetic radiation, regardless of frequency or angle of incidence. \\
Cool.\\
The blackbody spectrum roughly peaks at 
\[ f _ { \operatorname { man } } = ( 28 ) \frac { k T } { h } = \left( 5.9 \times 10 ^ { 10 } / \mathrm { K } \cdot \mathrm { s } \right) T \]
a result known as \textbf{\textit{Wein displacement law}}.\\
There is also a maximum $ \lambda_{max} $ (Note: that $ \lambda_{max} \neq c/f_{max} $)
\[ \lambda _ { \max } T = 2.9 \times 10 ^ { - 3 } \mathrm { m } \cdot \mathrm { K } \]
\indent This presence of such a maximum gives the prominent color to the radiation of a blackbody. For example, the Sun's surface is at some $ 6000 $K. The maximum wavelength is about $ 480 $nm, the middle range of the human eye, Coincident?! I think not! It actually is not. 
\section{Consequence of Particles as Waves}
Everything is basically a wave. You can use 
\[ \lambda = \dfrac{h}{p} \]
for anything, like dust particles, Ferraro, and ice cream. 
\section{The Nature of Photons+}
In the case of massless particles, like photons, energy is related as
\[ E = |\vec{p}| c \] 
We also know that Energy is related by Planck's constant and frequency
\[ E = hf \]
If we define angular frequency as $ \omega = 2 \pi f $ then we can also get energy as
\[ E = \hbar \omega \]
where \[ \hbar = h/2 \pi \]
\begin{example}
	Suppose that a 60 W lightbulb radiates primarily at a wavelength of 1000 nm, find the photons emitted per second. \\
	We use that 
	\[ f = c/\lambda \approx \scinot{3}{14} Hz \]
	so we can employ
	\begin{align*}
	n = \frac { 60 \mathrm { W } } { h f } = \frac { 60 \mathrm { W } } { \left( 6.63 \times 10 ^ { - 34 } \mathrm { J } \cdot \mathrm { s } \right) \left( 3 \times 10 ^ { 14 } \mathrm { s } ^ { - 1 } \right) } \\
	= 3 \times 10 ^ { 20 } \mathrm { photons/s }
	\end{align*}
\end{example}
\section{The Photoelectric Effect+}
If you strike a metal with a photon at a certain frequency (energy) then it will release electrons. done. \\
This is given by 
\[ \frac{1}{2} m_e v^2 = hf - W \]
where $ W $ is the work function of the metal. 
\begin{example}
	An experiment shows that wen EM radiation of $ \lambda 270$nm falls on an alunium surface, photoelectrons are emitted. The most energetic of these are stopped by a potential difference of $ 0.406 $V. Use this information to calculate the work function of alumninum in electron volts. \\
	The kinetic energy can be found as such
	\begin{align*}
	\begin{array} { c } { K = e V = \left( 1.6 \times 10 ^ { - 19 } \mathrm { C } \right) ( 0.406 \mathrm { V } ) = 0.65 \times 10 ^ { - 19 } \mathrm { J } } \\ { \text { The photon energy is } } \\ { E = h f = \frac { h c } { \lambda } = \left( 6.63 \times 10 ^ { - 34 } \mathrm { J } \cdot \mathrm { s } \right) \left( 3.00 \times 10 ^ { 8 } \mathrm { m } / \mathrm { s } \right) / \left( 270 \times 10 ^ { - 9 } \mathrm { m } \right) = 7.37 \times 10 ^ { - 19 } \mathrm { J } } \\ { \text { The difference is the work function: } } \\ { W = E - K = 6.72 \times 10 ^ { - 19 } \mathrm { J } = \left( 6.72 \times 10 ^ { - 19 } \mathrm { J } \right) / \left( 1.6 \times 10 ^ { - 19 } \mathrm { J } / \mathrm { eV } \right) = 4.2 \mathrm { eV } } 
	\end{array}
	\end{align*} 
\end{example}

\section{The Compton Effect+}
When a photon strikes a stationary electron then the photon and the electron are scattered and the transfer of energy result in a wavelength shift for the photon, which can be detected. 
\putpic{compton}
\\
We can solve for these values. Consider $ \textbf{q} $ the momentum of the photon before it strikes the electron. $ \textbf{q'} $ after it strike. And $ \textbf{p} $ momentum of the electron after being striked
\[ \textbf{q} = \textbf{q'} + \textbf{p} \]
and
\[ hf = m_ec^2 = hf' + \sqrt{p^2c^2 + m_e^2c^4} \]
solving for $ f $ and then $ \lambda $ we find that 
\[ \lambda' - \lambda = \frac{h}{m_2c}(1- \cos \theta) \]































