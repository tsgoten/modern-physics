\chapter{Lies From Classical Physics}
\section{Mechanics}
\textit{kinetic energy} is defined by \[ K = \dfrac{1}{2} mv^2 \]
and \textit{linear momentum} $ \vec{p} $ defined by \[ \vec{p} = m \vec{v} \]
In terms of the linear momentum, the kinetic energy can be written \[ K = \dfrac{p^2}{2m} \]
Corresponding to external force there is often a potential energy $ U $ such that (for one dimensional motion)
\[ F = -\dfrac{dU}{dx} \]
The total energy \[ E = K +U \]
When a particle moving with linear momentum $ \vec{p} $ is at a displacement $ \vec{r} $ from the origin $ O $, its \textit{angular momentum} $ \vec{L} $ about the point $ O $ is defined by \[ \vec{L} = \vec{r} \times \vec{p} \]
\section{Electricity and Magnetism}
The electrostatic force exerted by a charged particle $ q_1 $ on another charge $ q_2 $ has magnitude 
\[ F = \dfrac{1}{4 \pi \epsilon} \dfrac{|q_1||q_2|}{r^2} \] 
Using the previous relationship between potential energy and external force 
\[ U = \dfrac{1}{4 \pi \epsilon} \dfrac{q_1 q_2}{r} \] 
Electrostatic potential difference $ \Delta V $ is given by
\[ \Delta U = q \Delta V \]
For reference
\[ 1 eV = 1.602 \times 10^{-19}J \]
A magnetic field $ \vec{B} $ can be produced by current $ i $. The magnetic field through a loop of radius $ r $. 
\[ B = \dfrac{\mu_0 i}{2r} \]
\textit{magnetic moment} $ \vec{\mu} $ of a current loop:
\[ |\vec{\mu}| = iA\]
where $ A $ is the geometrical area enclosed by the loop. \\
When a current loop is placed in a uniform external magnetic field $ \vec{B_{ext}} $ factthere is a torque $ \vec{\tau} $ \[ \vec{\tau} = \vec{\mu} \times \vec{B_{ext}} \]
potential energy is given by
\[ U = - \vec{\mu} \centerdot \vec{B_{ext}} \]
Electromagnetic waves travel in free space with speed $ c $(speed of light), which is related to the electromagnetic constants $ \epsilon_0 $ and $ \mu_0 $
\[ c = (\epsilon_0 \mu_0)^{-1/2} \]
EM waves with frequency $ f $ and wavelength $ \lambda $ are related 
\[ c = \lambda f \]
\section{Kinetic Theory of Matter}
ideal gas equation 
\[ PV = NrT \]
The average kinetic energy of a molecule depends on its temperature \[ K_{av} = \dfrac{3}{2} kT \]
\section{Failures of Classical Concepts of Space and Time}
\subsection{Failure in Concept of Time}
Newton's laws are based on the assumption that time is the same, irrespective of the observer. This is shown to be incorrect.
\subsection{Failure in Concept of Space}
Galileo and Newton's theories are based on the assumption that space is the same, but this is not the case. 
\subsection{Failure in Concept of Velocity}
If space and time are not the same no matter what the same can be said for velocity, since velocity is dependent on space and time. \\
There is also a speed limit on the universe we call the \textit{speed of light} $ c $