\chapter{Wave Packets and the Uncertainty Principle}
\putpic{uncertainty-principle}
\section{A Free Electron in ONe Dimension}
Schrodinger Equation is
\[ i\hbar \dif{\wave}{t} = -\frac{h^2}{2m} \diff{}{x}{2} \wave + V(x)\wave\]
The time independent form wave equation is 
\[ \wave = u_E(x) \exp(\frac{-iEt}{\hbar}) \]
Since, Schrodinger's equation is a relationship between a function and it's second derivative we can try to express it like a spring function. the wave function as a second order differential equation of the form 
\[ f'' + k^2f = 0 \]
The Time-Independent Schrodinger Equation gives 
\[ Eu(x) = -\frac{\hbar^2}{2m} \diff{u(x)}{x}{2}\]
Which we can express in the form
\[ f'' + k^2 f = 0 \]
So, we find
\[ \diff{u(x)}{x}{2} + \frac{p^2}{\hbar^2} u(x) = 0  \]
Giving us
\[ k = \pm \frac{p}{\hbar} \]
or 
\[ p = \pm \hbar k \]
One important step is to normalize the wave function. That is 
\[ insert \]
\section{Wave Packets}
Solving for the probability distribution for the wave function gives 
\[ \Delta x = 4 \pi \hbar / \Delta p \]
the superposition of plane waves made up of weights that cover a finite range of momenta $ \Delta p $ is known as a wave packet. 
\section{Gaussian Wave Packet}
A wave packet without dispersion (real or imaginary part)
A wave packet with dispersion
In physics, a wave packet (or wave train) is a short "burst" or "envelope" of localized wave action that travels as a unit. 
\section{The Meaning of Uncertainty Relations}
\begin{example}
	Consider a grain of dust of mass $ \scinot{1}{-7} $ moving with a velocity around $ 10 m/s $ suppose that the measuring instrument has a local uncertainty of $ \scinot{1}{-6} m $, Find the intrinsic uncertainty in position. 
	\begin{align*}
	\begin{array} { c } { \Delta p = m \Delta v = \left( 10 ^ { - 7 } \mathrm { kg } \right) \left( 10 ^ { - 6 } \mathrm { m } / \mathrm { s } \right) = 10 ^ { - 13 } \mathrm { kg } \cdot \mathrm { m } / \mathrm { s } } \\ \intertext { Hence, according to the uncertainty relation, the position could at best be measured to within a window}\\ { \Delta x \equiv \frac { \hbar } { \Delta p } = \frac { \left( 1.05 \times 10 ^ { - 34 } \mathrm { J } \cdot \mathrm { s } \right) } { \left( 10 ^ { - 13 } \mathrm { kg } \cdot \mathrm { m } / \mathrm { s } \right) } \equiv 10 ^ { - 21 } \mathrm { m } } \end{array}
	\end{align*}
\end{example}
\section{The Time Energy Uncertainty Relation}
An extension of the previous uncertainty principle is time and energy. \\
\begin{align}
\begin{array} { c } { \Delta E = v \Delta p \geq v \frac { \hbar / 2 } { \Delta x } = \frac { \hbar / 2 } { ( \Delta x / v ) } = \frac { \hbar / 2 } { \Delta t } } \\ { \Delta E \Delta t \geq \hbar / 2 } \end{array}
\end{align}
Consider the case where the time interval is so short that the energy can be low enough to dissassociate and then come back together. 